\documentclass[9pt]{beamer}
\usetheme[faculty=fi]{fibeamer}
\usepackage[T2A]{fontenc}
\usepackage[utf8]{inputenc}
\usepackage{algpseudocode}
\usepackage{tikz}              % Для малювання діаграм
\usetikzlibrary{arrows.meta, positioning, shapes, calc} % TikZ бібліотеки
\usepackage[
  main=ukrainian, %% By using `czech` or `slovak` as the main locale
                %% instead of `english`, you can typeset the
                %% presentation in either Czech or Slovak,
                %% respectively.
  english, russian %% The additional keys allow foreign texts to be
]{babel}        %% typeset as follows:
%%
%%   \begin{otherlanguage}{czech}   ... \end{otherlanguage}
%%   \begin{otherlanguage}{slovak}  ... \end{otherlanguage}
%%
%% These macros specify information about the presentation
\title{Криптосистеми на еліптичних кривих} %% that will be typeset on the
%\subtitle{Presentation Subtitle}
\subtitle{Lecture 11: Isogenies}

\author{Грубіян Євген Олександрович}

%% These additional packages are used within the document:
\usepackage{ragged2e}  % `\justifying` text
\usepackage{booktabs}  % Tables
\usepackage{tabularx}
\usepackage{tikz}      % Diagrams
\usetikzlibrary{calc, shapes, backgrounds}
\usepackage{amsmath, amssymb}
\usepackage{url}       % `\url`s
\usepackage{listings}  % Code listings
\usepackage{wrapfig}
\frenchspacing
\begin{document}
  \frame{\maketitle}


  \begin{darkframes}
      
    \section{Light Frames}


% =============================================
\section{Поняття ізогенії}
% =============================================

% ---------------------------------------------

\begin{frame}{Поняття ізогенії}
  \frametitle{Поняття ізогенії}
  \begin{definition}
    Нехай $E_1$ та $E_2$ - еліптичні криві над полем $K$. \textbf{Ізогенія} $\phi: E_1 \to E_2$ - це скінченний морфізм кривих, що є також груповим гомоморфізмом, тобто $\phi(P+Q) = \phi(P) + \phi(Q)$ для всіх $P, Q \in E_1(\bar{K})$
  \end{definition}
  \begin{itemize}
    \item Якщо $\phi$ не нульовий морфізм (тобто не відображає всі точки в $\mathcal{O}_{E_2}$), то $\phi$ є сюр'єктивним відображенням.
    \item \textbf{Ядро ізогенії}: $\ker(\phi) = \{ P \in E_1(\bar{K}) \mid \phi(P) = \mathcal{O}_{E_2} \}$. Ядро завжди є скінченною підгрупою $E_1$.
    \item \textbf{Степінь ізогенії} $deg(\phi)$:
      \begin{itemize}
        \item Якщо $\phi$ сепарабельна, $deg(\phi) = |\ker(\phi)|$.
        \item Нульовий морфізм має степінь 0.
      \end{itemize}
    \item Якщо існує ненульова ізогенія $\phi: E_1 \to E_2$, криві $E_1$ та $E_2$ називаються \textbf{ізогенними}.
    \item Існує \textbf{дуальна ізогенія} $\hat{\phi}: E_2 \to E_1$ така, що $\hat{\phi} \circ \phi = [ deg(\phi) ]_{E_1}$ та $\phi \circ \hat{\phi} = [ deg(\phi) ]_{E_2}$, де $[m]$ - множення на $m$. Має той самий степінь: $deg(\hat{\phi}) = deg(\phi)$.
  \end{itemize}
\end{frame}

\begin{frame}{Теорема Тейта}
    \begin{block}{Теорема Тейта}
        Еліптичні криві $E/\mathbb{F}_q$ та $E'/\mathbb{F}_q$ є ізогенними тоді і тільки тоді коли $\#E(\mathbb{F}_q) = \#E'(\mathbb{F}_q)$
    \end{block}
    \begin{block}{Ізогенії суперсингулярних кривих}
        Якщо $E/\mathbb{F}_p$ - суперсингулярна, тоді $j(E)$ визначений в $\mathbb{F}_{p^2}$
    \end{block}
    Суперсингуляні криві мають ще одну дуже важливу особливість - кільце ендоморфізмів кривої $E/\mathbb{F}_q$: $\mathsf{End}(E)$ ізоморфне деякому ідеалу в алгебрі кватерніонів, тобто поза скалярним добутком $[m]:P \mapsto [m]P$ та ендоморфізмом Фробеніуса $\pi_q:(x,y) \mapsto(x^q,y^q)$ існують ще деякі 2 нетривіальні ендоморфізми(<<розмірність алгебри кватерніонів>> - 4). Тоді як в ординарному випадку $\pi_q,[m]$ - всі лінійно незалежні ендоморфізми що породжують $\mathsf{End}(E)$
\end{frame}

\begin{frame}{Приклади ізогеній}
  \frametitle{Приклади ізогеній}
  \begin{itemize}
    \item \textbf{Множення на ціле число $[m]$:}
      \begin{itemize}
        \item Для будь-якого цілого $m \neq 0$, відображення $\phi = [m]: E \to E$, що визначається як $P \mapsto P + \dots + P$ ($m$ разів), є ізогенією (ендоморфізмом).
        \item $\ker([m]) = E[m]$ - група точок $m$-кручення.
        \item $\deg([m]) = m^2$.
      \end{itemize}
    \item \textbf{Ендоморфізм Фробеніуса $\pi_q$ (над $\mathbb{F}_q$):}
      \begin{itemize}
        \item Для $E$ над $\mathbb{F}_q$, відображення $\pi_q: (x, y) \mapsto (x^q, y^q)$ є ендоморфізмом.
        \item $\ker(\pi_q - [1])$ = $E(\mathbb{F}_q)$.
        \item $\deg(\pi_q) = q$.
        \item $\pi_q$ є чисто несепарабельним.
      \end{itemize}
    \item \textbf{Ізогенія факторизації за підгрупою $E \to E/G$:}
      \begin{itemize}
        \item Нехай $G$ - скінченна підгрупа $E$. Існує (з точністю до ізоморфізму) єдина еліптична крива $E'$ та сепарабельна ізогенія $\phi: E \to E'$ така, що $\ker(\phi) = G$.
        \item Криву $E'$ позначають як $E/G$.
        \item $\deg(\phi) = |G|$.
      \end{itemize}
  \end{itemize}
\end{frame}

% ---------------------------------------------

\begin{frame}{Обчислення ізогеній: Формули Велу}
  \frametitle{Обчислення ізогеній: Формули Велу (Vélu)}
     Формули Велу (1971) надають явний алгоритм для обчислення ізогенії $\phi: E \to E/G$ та рівняння кривої $E/G$, коли задано криву $E$ та її скінченну підгрупу $G$.
    \begin{block}{Формули Велу}
        Нехай $G \subset E(\overline{\mathbb{F}_q})$ - підгрупа $E$. $E/\mathbb{F}_q: y^2=x^3 + ax +b$, $(E/G)/\mathbb{F}_q: y^2=x^3 + a'x +b'$ - еліптичні криві із сепарабельною ізогенією $\phi: E \to E/G$.
    \begin{align*} \phi(Q) &=( x(Q) + \sum_{P \in G \setminus \{\mathcal{O}\}} \left( x(Q+P) - x(P) \right), y(Q) + \sum_{P \in G \setminus \{\mathcal{O}\}} \left( y(Q+P) - y(P) \right) \\
    a'&=a-5\sum_{P \in G \setminus \{ \mathcal{O} \}} (3x(P)^2 + a), \quad b'= b-7\sum_{P \in G \setminus \{ \mathcal{O} \}} (5x(P)^3 + 3ax(P)+b)
    \end{align*}
    де $x(P)$, $y(P)$ - координати точки $P$.
    \end{block}
    % TODO: Навести конкретний приклад для G порядку 2 або 3, якщо дозволяє час/простір.
    \textbf{Важливо:} Формули Велу застосовні для будь-якої скінченної підгрупи $G$, але обчислення стають складними при зростанні $|G|$, тому зручно обчислювати ізогенії гладких порядків $|G|=l^e$ ітеративно
\end{frame}

\begin{frame}{Граф суперсингулярних $l$-ізогеній}
  \frametitle{Граф суперсингулярних $l$-ізогеній}
  Важливим об'єктом у сучасній криптографії на основі ізогеній є граф суперсингулярних ізогеній.

  \begin{itemize}
    \item \textbf{Вершини:} Множина класів ізоморфізму суперсингулярних еліптичних кривих, визначених над скінченним полем $\mathbb{F}_{p^2}$. Кожен клас ідентифікується своїм $j$-інваріантом. Кількість таких класів приблизно $p/12$.
    \item \textbf{Ребра:} Ізогенії малого простого порядку $l$ (наприклад, $l=2$ або $l=3$), що з'єднують ці криві. Якщо існує $l$-ізогенія $\phi: E_1 \to E_2$, то існує і дуальна $l$-ізогенія $\hat{\phi}: E_2 \to E_1$.
  \end{itemize}

  \textbf{Властивості графу:}
  \begin{itemize}
    \item \textbf{Зв'язність:} наслідок теореми Тейта.
    \item \textbf{Регулярність:} Граф є $(l+1)$-регулярним ($l \neq p$)
    \item \textbf{Експандер (Expander graph):} Для більшості параметрів $p$ та $l$ ці графи є експандерами. Це означає, що вони <<добре перемішані>>, і випадкові блукання швидко покривають граф.
    \item \textbf{Великий розмір:} Кількість вершин значна, що ускладнює повний перебір в задачі пошуку шляху(ізогенії) між двома вершинами (кривими)
  \end{itemize}
\end{frame}

\begin{frame}{Приклад: Локальна структура графу 3-ізогеній}
  \frametitle{Приклад: Локальна структура графу 3-ізогеній}
  \begin{center}
  \begin{tikzpicture}[
    scale=0.7, transform shape,
    curve_node/.style={circle, draw, thick, minimum size=25pt, inner sep=0pt},
    edge_label/.style={font=\scriptsize, midway, inner sep=1pt}
  ]
    % Central curve
    \node[curve_node] (E0) at (0,0) {$E_0$};

    % First level of 3-isogenous curves from E0
    \node[curve_node] (E1) at (-3,2.5) {$E_1$};
    \node[curve_node] (E2) at (3,2.5) {$E_2$};
    \node[curve_node] (E3) at (-3,-2.5) {$E_3$};
    \node[curve_node] (E4) at (3,-2.5) {$E_4$};

    \draw[-{Latex[length=2mm, width=1.5mm]}, thick] (E0) -- node[edge_label, sloped, above] {$\phi_1 (l=3)$} (E1);
    \draw[-{Latex[length=2mm, width=1.5mm]}, thick] (E0) -- node[edge_label, sloped, above] {$\phi_2 (l=3)$} (E2);
    \draw[-{Latex[length=2mm, width=1.5mm]}, thick] (E0) -- node[edge_label, sloped, below] {$\phi_3 (l=3)$} (E3);
    \draw[-{Latex[length=2mm, width=1.5mm]}, thick] (E0) -- node[edge_label, sloped, below] {$\phi_4 (l=3)$} (E4);

    % Second level of 3-isogenous curves from E1
    % One isogeny goes back to E0 (dual of phi1)
    \draw[-{Latex[length=2mm, width=1.5mm]}, thick, bend right=15] (E1) -- node[edge_label, sloped, below] {$\hat{\phi}_1 (l=3)$} (E0);

    % Other isogenies from E1
    \node[curve_node] (E11) at (-6,4) {$E_{11}$};
    \node[curve_node] (E12) at (-4.5,5) {$E_{12}$}; % Adjusted position for clarity
    \node[curve_node] (E13) at (-1.5,5) {$E_{13}$}; % Adjusted position for clarity

    \draw[-{Latex[length=2mm, width=1.5mm]}, thick] (E1) -- node[edge_label, sloped, above] {$\psi_1 (l=3)$} (E11);
    \draw[-{Latex[length=2mm, width=1.5mm]}, thick] (E1) -- node[edge_label, sloped, above] {$\psi_2 (l=3)$} (E12);
    \draw[-{Latex[length=2mm, width=1.5mm]}, thick] (E1) -- node[edge_label, sloped, above] {$\psi_3 (l=3)$} (E13);

    % Indication of continuation for other nodes
    \node[font=\Large] at ($(E2)+(1,0.5)$) {\ldots};
    \node[font=\Large] at ($(E3)+(1,-0.5)$) {\ldots};
    \node[font=\Large] at ($(E4)+(-1,-0.5)$) {\ldots};
    
    \node[below=3cm of E0, text width=10cm, align=center, font=\small]
    {Крива $E_0$ має 4 різні $3$-ізогенії ($\phi_1, \phi_2, \phi_3, \phi_4$) до кривих $E_1, E_2, E_3, E_4$.
    Крива $E_1$ також має 4 різні $3$-ізогенії: одна ($\hat{\phi}_1$) є дуальною до $\phi_1$ і веде назад до $E_0$,
    а інші ($\psi_1, \psi_2, \psi_3$) ведуть до нових кривих $E_{11}, E_{12}, E_{13}$.
    Аналогічна структура продовжується для всіх вершин графу.};

  \end{tikzpicture}
  \end{center}
\end{frame}

% =============================================
\section{Складні задачі та криптографія}
% =============================================

\begin{frame}{Складні обчислювальні задачі}
  \frametitle{Складні обчислювальні задачі на основі ізогеній}
  Припускається, що наступні задачі є обчислювально складними (в тому числі для квантових комп'ютерів):

  \begin{itemize}
    \item \textbf{Задача знаходження ізогенії (Isogeny Finding Problem):}
      Дано дві ізогенні еліптичні криві $E_1, E_2$. Знайти явну ізогенію $\phi: E_1 \to E_2$.
    \item \textbf{Задача обчислення ендоморфного кільця (Endomorphism Ring Computation):}
      Дано еліптичну криву $E$. Обчислити її кільце ендоморфізмів $\text{End}(E)$.
    \item \textbf{Задача обчислення шляху в графі ізогеній (Supersingular Isogeny Path Problem):}
      Дано дві суперсингулярні еліптичні криві $E_0, E_1$ та максимальний степінь $d$. Знайти послідовність ізогеній малих степенів $\phi_1, \dots, \phi_k$, що сполучає $E_0$ та $E_1$, де $deg(\phi_i) \le d$.
    \item \textbf{Суперсингулярна задача Діффі-Хеллмана (CSIDH/SSIDH Problem):}
      Дано суперсингулярну криву $E_0$ та образи $E_A = \phi_A(E_0)$, $E_B = \phi_B(E_0)$, де $\phi_A, \phi_B$ - невідомі (секретні) ізогенії з певними властивостями. Обчислити $j$-інваріант спільної кривої $E_{AB} \cong \phi_B(E_A) \cong \phi_A(E_B)$. (Примітка: задача для SIDH виявилася легшою через додаткові точки).
  \end{itemize}
  Ці задачі є основою для побудови постквантових криптосистем.
\end{frame}

% ---------------------------------------------

\begin{frame}{Криптографія на основі ізогеній}
  \frametitle{Криптографія на основі ізогеній}
  \begin{itemize}
    \item \textbf{Мотивація:} Пошук криптографічних систем, стійких до атак з використанням квантових комп'ютерів (постквантова криптографія, PQC).
    \item Алгоритм Шора (квантовий) ефективно розв'язує задачі факторизації та дискретного логарифмування (включаючи еліптичні криві), що робить RSA, DH, ECDH вразливими.
    \item Задачі, пов'язані з ізогеніями (особливо на суперсингулярних кривих), вважаються складними навіть для квантових комп'ютерів (хоча конкретні реалізації, як SIDH, можуть мати вразливості).
    \item \textbf{Переваги (потенційні):}
      \begin{itemize}
        \item Стійкість до квантових атак (для певних задач).
        \item Відносно малі розміри ключів та шифротекстів порівняно з іншими PQC кандидатами (наприклад, на основі решіток або кодів).
      \end{itemize}
    \item \textbf{Недоліки/Виклики:}
      \begin{itemize}
        \item Вища обчислювальна складність порівняно з класичною ECC.
        \item Менш досліджена безпека, недавні атаки на основні протоколи (SIDH).
      \end{itemize}
  \end{itemize}
\end{frame}

% =============================================
\section{Протокол SIDH}
% =============================================
\subsection{Налаштування та Ідея} % Додано підсекцію для кращої структури TOC

\begin{frame}{Протокол SIDH: Налаштування}
  \frametitle{Протокол SIDH: Налаштування (Setup)}
  \textbf{Публічні параметри:}
  \begin{itemize}
    \item Велике просте число $p$ спеціального вигляду: $p = l_A^{e_A} l_B^{e_B} f \pm 1$, де $l_A, l_B$ - малі різні прості числа, $e_A, e_B$ - великі показники, $f$ - малий кофактор.
    \item Поле $\mathbb{F}_{p^2}$.
    \item Стартова суперсингулярна еліптична крива $E_0$, визначена над $\mathbb{F}_{p^2}$. Структура групи: $E(\mathbb{F}_{p^2}) \cong (\mathbb{Z}/l_A^{e_A}\mathbb{Z})^2 \oplus (\mathbb{Z}/l_B^{e_B}\mathbb{Z})^2 \oplus (\mathbb{Z} / f\mathbb{Z})^2$
    \item Базисні точки для підгруп кручення:
      \begin{itemize}
          \item $\{P_A, Q_A\}$ - базис для $E_0[l_A^{e_A}] = \{ P \in E_0(\mathbb{F}_{p^2}) \mid [l_A^{e_A}]P = \mathcal{O} \}$.
          \item $\{P_B, Q_B\}$ - базис для $E_0[l_B^{e_B}] = \{ P \in E_0(\mathbb{F}_{p^2}) \mid [l_B^{e_B}]P = \mathcal{O} \}$.
      \end{itemize}
      Ці параметри є спільними та відомими всім учасникам.
  \end{itemize}
\end{frame}

% ---------------------------------------------

\begin{frame}{Протокол SIDH: Ідея обміну ключами}
 \frametitle{Протокол SIDH: Ідея обміну ключами}
  \begin{itemize}
      \item \textbf{Мета:} Аліса та Боб хочуть встановити спільний секретний ключ.
      \item \textbf{Секрети:}
          \item Аліса обирає секретну підгрупу $G_A \subset E_0[l_A^{e_A}]$ порядку $l_A^{e_A}$.
          \item Боб обирає секретну підгрупу $G_B \subset E_0[l_B^{e_B}]$ порядку $l_B^{e_B}$.
      \item \textbf{Ізогенії:}
          \item Аліса обчислює ізогенію $\phi_A: E_0 \to E_A = E_0 / G_A$.
          \item Боб обчислює ізогенію $\phi_B: E_0 \to E_B = E_0 / G_B$.
      \item \textbf{Обмін:} Вони обмінюються інформацією, яка дозволяє кожному обчислити образ секретної підгрупи іншого учасника на своїй кривій.
      \item \textbf{Спільний секрет:} Обидва обчислюють $j$-інваріант кривої $E_{AB} \cong E_0 / \langle G_A, G_B \rangle$. Завдяки властивостям ізогеній (комутативна діаграма), вони отримають однаковий результат: $j(\phi_B(E_A)) = j(\phi_A(E_B))$.
  \end{itemize}
  Далі розглянемо кроки детальніше.
\end{frame}

\subsection{Кроки Протоколу} % Додано підсекцію для кращої структури TOC

\begin{frame}{Протокол SIDH: Крок 1 (Аліса)}
  \frametitle{Протокол SIDH: Крок 1 (Секрет та дія Аліси)}
  \begin{enumerate}
    \item \textbf{Вибір секрету:} Аліса обирає два випадкових цілих числа $s_A, r_A \pmod{l_A^{e_A}}$ (не обидва нулі, часто $r_A=1$ і $s_A$ - секрет).
    \item \textbf{Формування ядра:} Аліса формує секретну точку (генератор ядра):
      $$ S_A = P_A + [s_A] Q_A \in E_0[l_A^{e_A}] $$
      Її секретна підгрупа $G_A = \langle S_A \rangle$. Це циклічна підгрупа порядку $l_A^{e_A}$.
      \textit{(Примітка: часто використовують випадкову точку $P_A + [s_A] Q_A$ як генератор)}
    \item \textbf{Обчислення ізогенії:} Аліса обчислює (за допомогою формул Велу) ізогенію $\phi_A: E_0 \to E_A = E_0 / G_A$ гладкого порядку $deg(\phi_A)=l_A^{e_A}$.
    \item \textbf{Обчислення образів точок Боба:} Аліса обчислює образи базисних точок Боба під дією своєї ізогенії:
      $ \phi_A(P_B) \quad \text{та} \quad \phi_A(Q_B) $.
      Ці точки лежать на кривій $E_A$ і мають порядок $l_B^{e_B}$.
    \item \textbf{Надсилання даних:} Аліса надсилає Бобу публічний ключ: $(j(E_A), \phi_A(P_B), \phi_A(Q_B))$.
  \end{enumerate}
\end{frame}

% ---------------------------------------------

\begin{frame}{Протокол SIDH: Крок 2 (Боб)}
  \frametitle{Протокол SIDH: Крок 2 (Секрет та дія Боба)}
  Аналогічно до Аліси:
  \begin{enumerate}
    \item \textbf{Вибір секрету:} Боб обирає два випадкових цілих числа $s_B, r_B \pmod{l_B^{e_B}}$ (не обидва нулі, часто $r_B=1$ і $s_B$ - секрет).
    \item \textbf{Формування ядра:} Боб формує секретну точку (генератор ядра):
      $$ S_B = P_B + [s_B] Q_B \in E_0[l_B^{e_B}] $$
      Його секретна підгрупа $G_B = \langle S_B \rangle$ - циклічна порядку $l_B^{e_B}$.
    \item \textbf{Обчислення ізогенії:} Боб обчислює ізогенію $\phi_B: E_0 \to E_B = E_0 / G_B$ гладкого порядку $deg(\phi_B)=l_B^{e_B}$.
    \item \textbf{Обчислення образів точок Аліси:} Боб обчислює образи базисних точок Аліси:
      $$ \phi_B(P_A) \quad \text{та} \quad \phi_B(Q_A) $$
      Ці точки лежать на кривій $E_B$ і мають порядок $l_A^{e_A}$.
    \item \textbf{Надсилання даних:} Боб надсилає Алісі публічний ключ: $(j(E_B), \phi_B(P_A), \phi_B(Q_A))$.
  \end{enumerate}
\end{frame}

% ---------------------------------------------

\begin{frame}{Протокол SIDH: Крок 3 (Обчислення секрету Алісою)}
  \frametitle{Протокол SIDH: Крок 3 (Обчислення секрету Алісою)}
  Аліса отримала $(E_B, \phi_B(P_A), \phi_B(Q_A))$ від Боба.
  \begin{enumerate}
    \item \textbf{Обчислення образу свого ядра на кривій Боба:} Аліса використовує свій секрет $s_A$ (або $S_A = P_A + [s_A] Q_A$) та отримані точки $\phi_B(P_A), \phi_B(Q_A)$, щоб обчислити точку на кривій $E_B$:
    $$ S'_A = \phi_B(S_A) = \phi_B(P_A + [s_A] Q_A) = \phi_B(P_A) + [s_A] \phi_B(Q_A) $$
    Ця точка $S'_A$ генерує підгрупу $\phi_B(G_A)$ порядку $l_A^{e_A}$ на кривій $E_B$.
    \item \textbf{Обчислення фінальної ізогенії:} Аліса обчислює ізогенію $\psi_A: E_B \to E_{BA}$, ядром якої є $\langle S'_A \rangle = \phi_B(G_A)$:
    $$ E_{BA} = E_B / \langle S'_A \rangle = E_B / \phi_B(G_A) $$
    \item \textbf{Спільний секрет:} Аліса обчислює $j$-інваріант кривої $E_{BA}$.
    $$ k_A = j(E_{BA}) $$
    Це її версія спільного секрету.
  \end{enumerate}
\end{frame}

% ---------------------------------------------

\begin{frame}{Протокол SIDH: Крок 4 (Обчислення секрету Бобом)}
  \frametitle{Протокол SIDH: Крок 4 (Обчислення секрету Бобом)}
  Боб отримав $(E_A, \phi_A(P_B), \phi_A(Q_B))$ від Аліси.
  \begin{enumerate}
    \item \textbf{Обчислення образу свого ядра на кривій Аліси:} Боб використовує свій секрет $s_B$ (або $S_B = P_B + [s_B] Q_B$) та отримані точки $\phi_A(P_B), \phi_A(Q_B)$, щоб обчислити точку на кривій $E_A$:
    $$ S'_B = \phi_A(S_B) = \phi_A(P_B + [s_B] Q_B) = \phi_A(P_B) + [s_B] \phi_A(Q_B) $$
    Ця точка $S'_B$ генерує підгрупу $\phi_A(G_B)$ порядку $l_B^{e_B}$ на кривій $E_A$.
    \item \textbf{Обчислення фінальної ізогенії:} Боб обчислює ізогенію $\psi_B: E_A \to E_{AB}$, ядром якої є $\langle S'_B \rangle = \phi_A(G_B)$:
    $$ E_{AB} = E_A / \langle S'_B \rangle = E_A / \phi_A(G_B) $$
    \item \textbf{Спільний секрет:} Боб обчислює $j$-інваріант кривої $E_{AB}$.
    $$ k_B = j(E_{AB}) $$
    Це його версія спільного секрету.
  \end{enumerate}
\end{frame}

% ---------------------------------------------

\begin{frame}{Протокол SIDH: Комутативність та Секрет}
 \frametitle{Протокол SIDH: Комутативність та Спільний Секрет}
 \begin{itemize}
     \item \textbf{Ключова властивість:} Ізогенії $\phi_A$ та $\phi_B$ мають ядра з порядками, що є взаємно простими ($l_A^{e_A}$ та $l_B^{e_B}$). Це призводить до "комутативності" діаграми (з точністю до ізоморфізму):
     $$ E_{BA} = E_B / \phi_B(G_A) \cong E_0 / \langle G_A, G_B \rangle $$
     $$ E_{AB} = E_A / \phi_A(G_B) \cong E_0 / \langle G_A, G_B \rangle $$
     Тому $E_{AB}$, $E_{BA}$ є ізоморфними над $\mathbb{F}_{p^2}$.
     \item \textbf{Спільний секрет:} Ізоморфні криві мають однаковий $j$-інваріант. Отже, Аліса та Боб обчислюють спільний секрет:
     $$ k_A = j(E_{BA}) = j(E_{AB}) = k_B $$
     \item \textbf{Навіщо потрібні $\phi(P), \phi(Q)$?} Обчислення $\phi_A(P_B), \phi_A(Q_B)$ (і аналогічно для Боба) є критичним. Воно дозволяє стороні обчислити образ ядра іншої сторони на \textit{своїй} проміжній кривій ($E_A$ або $E_B$), не знаючи секретної ізогенії іншої сторони. Саме ці "додаткові точки" (auxiliary points) стали вектором атаки Кастрика-Декру.
 \end{itemize}
\end{frame}

% ---------------------------------------------

\begin{frame}{Протокол SIDH: Візуалізація (TikZ)}
  \frametitle{Протокол SIDH: Візуалізація}

  \begin{tikzpicture}[
    scale=1,
    node distance=2cm and 2.5cm,
    curve/.style={draw, ellipse, minimum height=1cm, minimum width=1.5cm, align=center},
    arrow/.style={-latex, thick},
    label/.style={midway, inner sep=1pt, font=\footnotesize}
  ]

    % Nodes for curves
    \node[curve] (E0) {$E_0$};
    \node[curve, align=center, below left=of E0] (EA) {$E_A = E_0/G_A$};
    \node[curve, align=center, below right=of E0] (EB) {$E_B = E_0/G_B$};
    \node[curve, align=center, below right=of EA] (EAB) {$E_{AB} \cong E_{BA}$ \\ ($j(E_{AB})$ - секрет)};

    % Arrows for isogenies
    \path[arrow] (E0) edge node[label, left] {$\phi_A$ (секрет Аліси $G_A$)} (EA);
    \path[arrow] (E0) edge node[label, right] {$\phi_B$ (секрет Боба $G_B$)} (EB);
    \path[arrow] (EA) edge node[label, left] {$\psi_B$ (ядро $\phi_A(G_B)$)} (EAB);
    \path[arrow] (EB) edge node[label, left] {$\psi_A$ (ядро $\phi_B(G_A)$)} (EAB);

    % Annotations (optional)
    \node[below=0.2cm of EA, text width=3cm, align=center, font=\footnotesize] {Аліса обчислює $E_A$ та $\phi_A(P_B), \phi_A(Q_B)$};
    \node[below=0.2cm of EB, text width=3cm, align=center, font=\footnotesize] {Боб обчислює $E_B$ та $\phi_B(P_A), \phi_B(Q_A)$};
    \node[above=0.8cm of EAB, text width=3cm, align=center, font=\footnotesize] {Аліса: обчислює $E_{BA}$ з $E_B$ і $S'_A=\phi_B(S_A)$. \\ Боб: обчислює $E_{AB}$ з $E_A$ і $S'_B=\phi_A(S_B)$.};

  \end{tikzpicture}
  
\end{frame}

% ---------------------------------------------

\begin{frame}{Протокол SIKE}
  \frametitle{Протокол SIKE (Supersingular Isogeny Key Encapsulation)}
  \begin{itemize}
    \item Механізм інкапсуляції ключів (KEM), побудований на основі SIDH.
    \item Був кандидатом у 3-му та фінальному (4-му) раундах конкурсу постквантової криптографії NIST PQC.
    \item Використовував перетворення типу Фудзісакі-Окамото для перетворення схеми обміну ключами SIDH (типу PKE) на безпечний KEM (стійкий до атак типу CCA2).
    \item \textbf{Основна ідея KEM:}
      \begin{itemize}
        \item \textbf{KeyGen:} Генерує пару ключів (відкритий $pk$, секретний $sk$), аналогічно до SIDH.
        \item \textbf{Encaps:} Бере відкритий ключ $pk$ отримувача. Генерує випадковий спільний секрет $K$ та його інкапсуляцію (шифротекст) $C$. Надсилає $C$ отримувачу.
        \item \textbf{Decaps:} Отримувач використовує свій секретний ключ $sk$ та отриманий шифротекст $C$ для відновлення того ж спільного секрету $K$.
      \end{itemize}
    \item Розглядався як один із найперспективніших кандидатів PQC через малі розміри ключів.
    \item Став вразливим через атаку на базовий протокол SIDH.
  \end{itemize}
\end{frame}


% =============================================
\section{Атака на SIDH та сучасні алгоритми}
% =============================================

\begin{frame}{Атака на SIDH/SIKE (2022)}
  \frametitle{Атака на SIDH/SIKE (Castryck-Decru, 2022)}
  \begin{itemize}
    \item У серпні 2022 року Воутер Кастрик (Wouter Castryck) та Тома Декру (Thomas Decru) представили ефективну атаку на протокол SIDH та, як наслідок, на SIKE.
    \item \textbf{Ключовий момент:} Атака використовує саме ті \textit{додаткові точки кручення} ($\phi_A(P_B), \phi_A(Q_B)$ та $\phi_B(P_A), \phi_B(Q_A)$), які передаються в протоколі SIDH.
    \item \textbf{Ідея атаки (дуже спрощено):}
      \begin{itemize}
        \item Використовує зв'язок між ізогеніями еліптичних кривих та ізогеніями абелевих многовидів вищих розмірностей (зокрема, поверхнями Куммера, пов'язаними з якобіанами гіпереліптичних кривих роду 2).
        \item Знання образів додаткових точок дозволяє ефективно відновити інформацію про секретну ізогенію (її ядро, або еквівалентно, секретний скаляр $s_A$ чи $s_B$).
        \item Алгоритм використовує так звані ізогенії Рішело (Richelot isogenies) між якобіанами кривих роду 2, які можна побудувати за допомогою інформації з SIDH.
      \end{itemize}
    \item \textbf{Результат:} Атака дозволяє відновити секретний ключ Аліси або Боба за поліноміальний час на класичному комп'ютері.
    \item Це повністю зламало безпеку SIDH та SIKE у їхній відомій формі.
  \end{itemize}
\end{frame}

% ---------------------------------------------

\begin{frame}{Наслідки атаки та сучасний стан}
  \frametitle{Наслідки атаки та сучасний стан}
  \begin{itemize}
    \item \textbf{SIKE відкликано:} NIST відкликав SIKE зі списку кандидатів PQC для стандартизації невдовзі після публікації атаки.
    \item \textbf{Пошук нових підходів:} Атака стимулювала пошук нових криптографічних схем на основі ізогеній, які б не використовували додаткові точки кручення у такий самий спосіб, або базувалися б на інших варіантах задачі ізогеній (наприклад, CSIDH, яке не використовує $\mathbb{F}_{p^2}$ та додаткові точки).
    \item \textbf{Активна область досліджень:} Криптографія на основі ізогеній залишається активною, хоча й складнішою, областю досліджень.
    \item \textbf{Альтернативні задачі:} Розглядаються схеми, що базуються на:
      \begin{itemize}
        \item Задачі знаходження шляху в графі суперсингулярних ізогеній (без додаткових точок, як у CSIDH).
        \item Задачі обчислення ендоморфного кільця.
        \item Інших варіантах задачі обчислення ізогеній.
      \end{itemize}
  \end{itemize}
\end{frame}

% ---------------------------------------------

\begin{frame}{Сучасні алгоритми: SKISign}
  \frametitle{Сучасні алгоритми: SKISign та SQISign}
  \begin{itemize}
    \item \textbf{SKISign (Small Key Isogeny Signature) / SQISign (Short Quaternion Isogeny Signature):} Схеми цифрового підпису на основі ізогеній. SQISign є новішою та ефективнішою версією.
    \item \textbf{Інша базова задача:} Безпека базується на складності задачі знаходження ізогенійного шляху між двома заданими суперсингулярними кривими (Supersingular Isogeny Path Problem) та пов'язаних задачах у графі ізогеній. SQISign також використовує структуру кватерніонних алгебр.
    \item \textbf{Не використовує SIDH-структуру:} Схеми побудовані інакше і не використовують обмін додатковими точками, як у SIDH. Тому атака Кастрика-Декру на них безпосередньо не застосовується.
    \item \textbf{Ідея підпису (GPS/Fiat-Shamir):} Схема використовує підхід типу доказу з нульовим розголошенням (перетворений на підпис за допомогою Fiat-Shamir), де доказом знання секрету (секретного шляху ізогеній) є здатність відповісти на криптографічний виклик.
    \item \textbf{Переваги:} Постквантова стійкість, дуже малі розміри підписів (особливо SQISign) порівняно з іншими PQC підписами.
    \item \textbf{Недоліки:} Досить повільна генерація та перевірка підпису. Параметри та безпека все ще потребують ретельного аналізу.
    \item SQISign розглядається як один із найперспективніших напрямків у постквантових підписах на основі ізогеній.
  \end{itemize}
\end{frame}

% =============================================
\section{Висновки}
% =============================================

\begin{frame}{Висновки}
  \frametitle{Висновки}
  \begin{itemize}
    \item Ізогенії є фундаментальними об'єктами в теорії еліптичних кривих, що описують структурні зв'язки між ними.
    \item Формули Велу дозволяють явно обчислювати ізогенії за їхніми ядрами.
    \item Складність обчислення ізогеній (особливо між суперсингулярними кривими) стала основою для розробки постквантових криптосистем.
    \item Протоколи SIDH/SIKE були перспективними, але виявилися вразливими через використання додаткової інформації (точок кручення).
    \item Атака 2022 року стала важливим уроком для спільноти PQC.
    \item Дослідження продовжуються, фокусуючись на альтернативних задачах та конструкціях (CSIDH, SQISign), що демонструє життєздатність напрямку, хоч і з новими викликами.
  \end{itemize}
\end{frame}

  \end{darkframes}




\end{document}
