\documentclass[9pt]{beamer}
\usetheme[faculty=fi]{fibeamer}
\usepackage[T2A]{fontenc}
\usepackage[utf8]{inputenc}
\usepackage{algpseudocode}
\usepackage[
  main=ukrainian, %% By using `czech` or `slovak` as the main locale
                %% instead of `english`, you can typeset the
                %% presentation in either Czech or Slovak,
                %% respectively.
  english, russian %% The additional keys allow foreign texts to be
]{babel}        %% typeset as follows:
%%
%%   \begin{otherlanguage}{czech}   ... \end{otherlanguage}
%%   \begin{otherlanguage}{slovak}  ... \end{otherlanguage}
%%
%% These macros specify information about the presentation
\title{Криптосистеми на еліптичних кривих} %% that will be typeset on the
%\subtitle{Presentation Subtitle}
\subtitle{Lecture 10: Pairings Cryptography}

\author{Грубіян Євген Олександрович}

%% These additional packages are used within the document:
\usepackage{ragged2e}  % `\justifying` text
\usepackage{booktabs}  % Tables
\usepackage{tabularx}
\usepackage{tikz}      % Diagrams
\usetikzlibrary{calc, shapes, backgrounds}
\usepackage{amsmath, amssymb}
\usepackage{url}       % `\url`s
\usepackage{listings}  % Code listings
\usepackage{wrapfig}
\frenchspacing
\begin{document}
  \frame{\maketitle}


  \begin{darkframes}
      
    \section{Light Frames}

\section{Типи білінійних спарювань}

\begin{frame}{Степінь вкладення}
    Нам відомо що структура групи кручення: $E[n] \cong Z_n \times Z_n$ якщо $q \nmid n$. Тобто $\#E[n]=n^2$
    
    Нагадаємо що степінь вкладення ЕК $E/F_q, E(F_q)=r$ в поле $F_q$ - таке мінімальне значення $k$ що $r | q^k - 1$.

    Нехай спарювання Вейля визначене $e: E[r] \times E[r] \to F_{q^k}^*$. 
    \begin{block}{Теорема}
    Якщо $k$ - степінь вкладення $E/F_q$ то $E[r] \subseteq E(F_{q^k})$. 
    \end{block}
    На практиці нам потрібні не тільки ефективно обчислювати значення спарювання, але й ефективно гешувати в елементи групи (обирати випадкові точки наприклад) та відображати елементи між групами. До того ж порядок груп має бути простим. Тому більш специфічно можна визначити спарювання як:
    $$e: \mathbb{G}_1 \times \mathbb{G}_2 \to F_{q^k}^*$$
    Де $\mathbb{G}_1, \mathbb{G}_2$ деякі підгрупи $E[r]$.
\end{frame}

\begin{frame}{Структура групи: приклад}
\begin{columns}
        \begin{column}{0.5\textwidth}
        \begin{block}{Теорема}
            $$ \#E(F_{q^k}) = q^k +1 -\alpha^k -\beta^k $$ 
            Де $\alpha, \beta$ - корені характеристичного рівняння $x^2 -tx + q$ в $F_{q^k}$($t$ - слід ендоморфізму Фробеніуса)
        \end{block}
Візьмемо суперсингулярну криву $E/F_{11}:y^2=x^3+4, E(F_{11})=12, E(F_{11^2})=144$. Бачимо що $r \nmid q-1$, але $r \mid q^2-1$. тому степінь вкладення $k=2$. Таким чином $E[3] \cong Z_3 \times Z_3$ тобто $\#E[3]=9$ та ми маємо 4 циклічні підгрупи порядку 3:
        \end{column}

        \begin{column}{0.5\textwidth}
            \begin{figure}
        \centering
        \includegraphics[width=1\linewidth]{resources/torsion.png}
        \label{fig:enter-label}
    \end{figure}
        \end{column}
        \end{columns}
\end{frame}
\begin{frame}{Вибір підгруп для спарювань}
\begin{block}{Слід точки $P$}
    $$Tr(P) = \sum_{i=0}^{k-1} \pi^{i} (P) = \sum_{i=0}^{k-1}(x^{q^i},y^{q^i})$$ 
    Де $\pi:\pi((x,y))=(x^q,y^q)$ - ендоморфізм Фробеніуса.
\end{block}
\begin{block}{Підгрупа $\mathcal{G}_1 \subset E[r]$}
    $$\mathcal{G}_1 = E[r] \cap ker(\pi - [1])= \{ P \in E[r] | Tr(P) \in \mathcal{G}_1 \}$$
    Ця група співпадає із $E(F_q)$, оскільки $\forall P \in \mathcal{G}_1: Tr(P) = [k]P$.
\end{block}

\begin{block}{Підгрупа сліду 0: $\mathcal{G}_2 \subset E[r]$}
    $$\mathcal{G}_2 = E[r] \cap ker(\pi - [q])= \{ P \in E[r] | Tr(P) = \mathcal{O}_E \} $$
    Це єдина підгрупа в $E[r]$, для якої $\forall P \in E[r]: Tr(P)=\mathcal{O}$
\end{block}

\end{frame}

\begin{frame}{Структура підгруп}
    \begin{block}{Анти-слід точки $P$}
        $$aTr(P) = [k]P - Tr(P)$$
        Зауважимо що $\forall P \in E[r]:aTr(P) \in \mathcal{G}_2$
    \end{block}
    \begin{block}{Теорема}
        Нехай $P_1 \in \mathcal{G}_1, P_2 \in \mathcal{G}_2$, тоді $\forall P \in E[r] \exists i,j \in Z_q^*: P=[i]P_1 +[j]P_2$
    \end{block}
    Таким чином $\mathcal{G}_1$ та $\mathcal{G}_2$ свого роду утворюють базис групи $E[r]$ як <<вектороного простору розмірності 2>> (група $E[r]$ є модулем над $Z_q$).

    Тому якщо в якості $\mathbb{G}_1$ взяти $\mathcal{G}_1$, а $\mathbb{G}_2 = \mathcal{G}_2$ то спарювання Вейля точно буде не виродженим: $\forall P \in \mathcal{G}_1,Q \in \mathcal{G}_2: e(P,Q) \neq 1$! Зауважимо якщо $\mathbb{G}_1 =\mathbb{G}_2 = \mathcal{G}_1$ то відображення буде виродженим за властивостями спарювання Вейля. Хоч такий вибір параметрів найбільш поширений на практиці (так званий Type 3), але такий підхід має ряд недоліків, зокрема відсутність жодного ефективно-обчислюваного відображення із $\mathcal{G}_2$ в $\mathcal{G}_1$.
\end{frame}

\begin{frame}{Діграма підгруп}
    \begin{figure}
        \centering
        \includegraphics[width=0.9\linewidth]{resources/aTr.png}
        \label{fig:enter-label}
    \end{figure}
\end{frame}

\begin{frame}{Типи спарювань}
    \begin{block}{Відображення розсіювання}
        Якщо $E/F_q$ - суперсингулярна, тоді існує відображення розсіювання $\psi:E(F_q) \to E(F_{q^k})$. Зокрема якщо $k=2$ тоді $\psi (x,y)=(-x,iy)$.
    \end{block}
    На практиці виділяють 4 типи білінійних спарювань на еліптичних кривих. У всіх 4х типах зручно взяти $\mathbb{G}_1 = \mathcal{G}_1$. Таким чином вибір $\mathbb{G}_2$ буде визначати тип.
      \begin{description}
    \item[Type\,1:] $\mathbb{G}_1=\mathbb{G}_2=\mathcal{G}_1$, при чому $E/F_q$ - суперсингулярна, тобто $\#E(F_q) = q+1$. В такому випадку $e(P,Q)=e_w(P,\psi(Q))$. Основним недоліком цього підходу є ефективність, оскільки суперсингулярні криві мають як правило значно більший розмір поля для забезпечення рівня стійкості.
    \item[Type\,2:] $\mathbb{G}_1=\mathcal{G}_1, \mathbb{G}_2$ - довільна підгрупа що не співпадає із $\mathbb{G}_1$. Осеновний недолік - неможливість ефективно гешувати в елементи $\mathbb{G}_2$, що унеможливлює застосування в ряді протоколів.
    
  \end{description}
\end{frame}
\begin{frame}{Типи спарювань}
    \begin{figure}
        \centering
        \includegraphics[width=0.7\linewidth]{resources/twist.png}
        \label{fig:enter-label}
        \caption{$E$ та її квадратичне кручення $E'$}
    \end{figure}
    \begin{description}
        \item[Type\,3:] $\mathbb{G}_1 = \mathcal{G}_1, \mathbb{G}_2=\mathcal{G}_2$. Найбільш поширений та ефективний вид спарювань (зокрема через змогу застосовувати криві кручення для прискорення). Можна ефективно гешувати в елементи $\mathbb{G}_2$ за допомогою $aTr$. Основний недолік - відсутність ефективних відображень $\mathbb{G}_2 \to \mathbb{G}_1$. 
        \item[Type\,4:] $\mathbb{G}_1 = \mathcal{G}_1, \mathbb{G}_2 = E[r]$. Гешування в $\mathbb{G}_2$ можливе, але не настільки легке, також ефективність даного типу програє порівняно з третім
    \end{description}
\end{frame}

\begin{frame}{Спарювання в криптографії}
    Зазвичай спарювання застосовують в різноманітних протоколах де задача CDH(Computational Diffie-Hellman) вважається складною. 
    
    Також додатково вводять припущення безпеки на саме спарювання, до прикладу \textit{The fixed argument pairing inversion problem (FAPI-1)} - маючи $P \in \mathbb{G}_1,g \in F_{q^k}$ обчислити $Q \in \mathbb{G}_2$ таке що $e(P,Q)=g$.

    На практиці дизайн протоколу має передбачати ефективність обчислення $e$, стійкість до DLP в $E(F_{q^k})$ зокрема із врахуванням MOV-атаки(що майже неодмінно веде до росту розміру підгрупи або степеня вкладення), такі криві називаються \textit{pairing-friendly}.
    \begin{block}{Barreto-Naehrig curves}
        Криві $E/F_p:y^2=x^3+b$, де $p=36x^4+36x^3+24x^2+6x+1$ - просте для деякого $x$ та відповідно підібраного генератора $F_p^*$ $b$. Тоді $r=36x^4+36x^3+18x^2+6x+1$, а степінь вкладення $k=12$. Знаменита крива BN254 задається $x=4965661367192848881, b=2$.
    \end{block}
\end{frame}
\section{Перевірка квадратичних рівнянь}
\begin{frame}{Спарювання для перевірки квадратичних рівнянь}
  Зазначимо що перевіряти лінійні співвідношення на точках еліптичної кривої досить легко застосовуючи звичайну арифметику: тобто якщо $x_1+x_2=c$ то $[x_1]P+[x_2]P=[c]P$
  
  Ускладнимо задачу: нехай Аліса має переконати Боба що вона знає деякі $x_1,x_2$ що добуток $x_1x_2 = c$, але при цьому не розкриваючи значень $x_1,x_2$ Бобу. 
  \begin{block}{Протокол перевірки квадратичності}
      \begin{enumerate}
          \item Аліса обирає $P \in E[r]$, обчислює $[x_1]P, [x_2]P$ та надсилає їх Бобу
          \item Боб перевіряє $e([x_1]P, [x_2]P)=e(P,P)^c$
      \end{enumerate}
    Зауваження: для спрощення наведено спарювання першого типу, але нічого не заважає узагальнити протокол на довільне.
  \end{block}

  Взагалі кажучи цей протокол можна поширити на довільну арифметичну схему (QAP), тобто довести що ми коректно обчислили значення складного арифметичного виразу без розкриття його входів.
\end{frame}

\section{BLS-підпис}
\begin{frame}{Криптосистема BLS-підпису}
\begin{block}{Підпис BLS(Boneh–Lynn–Shacham)}
  Сторони узгоджують параметри: спарювання $e:\mathbb{G}_1 \times \mathbb{G_2} \to \mathbb{G}_T$, генератори $G_1 \in \mathbb{G}_1, G_2 \in \mathbb{G}_2$, геш функцію $H: \mathcal{M} \to \mathbb{G}_1$
  \begin{description}
      \item[$KeyGen(1^\lambda) \to (sk, pk)$] $sk \gets Z_r, pk \gets [sk]G_2$
      \item[$Sign(m) \to \sigma$] $\sigma \gets [sk]H(m) \in \mathbb{G}_1$
      \item[$Verify(\sigma, m) \to bool$] $e(H(m),pk) =^? (\sigma, G_2)$
  \end{description}
  \end{block}
  З плюсів: криптосистема підтримує можливість ефективно агрегувати та перевіряти підписи різних підписників на різні повідомлення. 
\end{frame}

\begin{frame}{Криптосистема шифрування на ідентифікаторах IBE (Boneh-Franklin)}
    Перша криптосистема шифрування на основі ідентифікаторів із застосуванням білінійного спарювання, що запропонована в 2001р Боне та Франкліном.
    
    Сторони узгоджують параметри: спарювання $e:\mathbb{G}_1 \times \mathbb{G}_1 \to \mathbb{G}_T$, генератор $G_1 \in \mathbb{G}_1$, геш функцію $H_1: \{0,1\}^* \to \mathbb{G}_1, H_2: \mathbb{G}_2 \to \mathcal{M}$.
      \begin{description}
      \item[$KeyGen(1^\lambda) \to (mk, P_{pub})$] $mk \gets Z_q, P_{pub}=[mk]P$
      \item[$Extract(mk, ID) \to (sk_{ID}, Q_{ID})$] $Q_{ID} = H_1(ID), sk_{ID}=[mk]Q_{ID}$
      \item[$Encrypt(m, P_{pub}, ID) \to (U,V)$] $t \xleftarrow{R}Z_r, U \gets [t]P_1, V=M \oplus H_2(e(Q_{ID}, P_{pub})^t)$
      \item[$Decrypt(U,V) \to M$] $M \gets V \oplus H_2(e(sk_{ID},U))$
  \end{description}
\end{frame}




  \end{darkframes}




\end{document}
