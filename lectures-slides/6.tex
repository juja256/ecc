\documentclass[9pt]{beamer}
\usetheme[faculty=fi]{fibeamer}
\usepackage[T2A]{fontenc}
\usepackage[utf8]{inputenc}
\usepackage[
  main=ukrainian, %% By using `czech` or `slovak` as the main locale
                %% instead of `english`, you can typeset the
                %% presentation in either Czech or Slovak,
                %% respectively.
  english, russian %% The additional keys allow foreign texts to be
]{babel}        %% typeset as follows:
%%
%%   \begin{otherlanguage}{czech}   ... \end{otherlanguage}
%%   \begin{otherlanguage}{slovak}  ... \end{otherlanguage}
%%
%% These macros specify information about the presentation
\title{Криптосистеми на еліптичних кривих} %% that will be typeset on the
%\subtitle{Presentation Subtitle}
\subtitle{Lecture 6: Classic Protocols}

\author{Грубіян Євген Олександрович}

%% These additional packages are used within the document:
\usepackage{ragged2e}  % `\justifying` text
\usepackage{booktabs}  % Tables
\usepackage{tabularx}
\usepackage{tikz}      % Diagrams
\usetikzlibrary{calc, shapes, backgrounds}
\usepackage{amsmath, amssymb}
\usepackage{url}       % `\url`s
\usepackage{listings}  % Code listings
\usepackage{wrapfig}
\frenchspacing
\begin{document}
  \frame{\maketitle}


  \begin{darkframes}
      
    \section{Light Frames}



%%%%%%%%%%%%%%%%%%%%%%%%%%%%%%%%%%%%%%%%%%%%%%%%%%%%%%%%%%%%%%%%%%%%%%
% Слайд 1: Мотивація та загальний огляд
%%%%%%%%%%%%%%%%%%%%%%%%%%%%%%%%%%%%%%%%%%%%%%%%%%%%%%%%%%%%%%%%%%%%%%
\begin{frame}{Криптосистеми на еліптичних кривих}
  \begin{itemize}
    \item \textbf{Мотивація:} Стійкість сучасних криптосистем базується на складності задачі дискретного логарифму в групах точок еліптичних кривих.
    \item \textbf{Основні протоколи:}
      \begin{itemize}
        \item Протокол обміну ключами (ECDH)
        \item Шифрування Ель-Гамаля та інкапсуляція ключа
        \item Підпис ECDSA
        \item Схеми ідентифікації та підпису Шнорра
        \item Криптографічні комітменти (Педерсен)
      \end{itemize}
  \end{itemize}
  Вибір параметрів для протоколів
  \begin{itemize}
      \item Обираємо(але обережено, з деякими виключеннями) еліптичну криву над полем $F_p$: $E/F_p$, так що $|E(F_p)|=nl$, де $n$-велике просте число, $l$-малий кофактор. 
      \item В якості основної групи $G$ для протоколів беруть просту підгрупу порядку $q$
      \item Обираємо генератор групи деяку точку $P \in G$
  \end{itemize}
\end{frame}

%%%%%%%%%%%%%%%%%%%%%%%%%%%%%%%%%%%%%%%%%%%%%%%%%%%%%%%%%%%%%%%%%%%%%%
% Слайд 2: ECDH та шифрування Ель-Гамаля з моделлю стійкості
%%%%%%%%%%%%%%%%%%%%%%%%%%%%%%%%%%%%%%%%%%%%%%%%%%%%%%%%%%%%%%%%%%%%%%
\begin{frame}{Обмін ключами ECDH}
  \textbf{ECDH (обмін ключами):}
  \begin{enumerate}
    \item Аліса і Боб обирають секрети (наприклад, \(a\) і \(b\)).
    \item Обмінюються публічними значеннями: \(Q_A = [a]P\), \(Q_B = [b]P\).
    \item Обчислюють спільний секрет: \(S = [a]Q_B = [b]Q_A\)
  \end{enumerate}
    Можливі атаки третього посередині (Man-in-The-Middle), а також CCA (Static-Diffie-Hellman attack by Brown, Galant 2004), тому в такому вигляді застосовувати $ECDH$ не можна. 
    
    Існують модифікації протоколу, які передбачають автентифікацію та захищають від ряду атак, зокрема $3XDH$ від Signal:
    \begin{itemize}
        \item Аліса та Боб мають пари довгостроковий (ідентифікаційний) та короткостроковий (ефемерний) ключів: \begin{align*}
            (IK_A=[ik_a]P, EK_A=[ek_a]P) \rightarrow \\
            \leftarrow & (IK_B=[ik_B]P, EK_B=[ek_B]P)
        \end{align*}
        \item Спільний секрет $S=KDF([ik_a]EK_B, [ek_a]IK_B, [ek_a]EK_B)$
    \end{itemize}
\end{frame}

\begin{frame}{Інкапсуляція ключа ECDH-KEM}
      \textbf{Проста KEM-схема на основі ECDH (RFC 5753):}
  \begin{itemize}
  \item $KeyGen(1^\lambda) \rightarrow (x_B, Q_B)$, $x_B$ - секретний ключ, $Q_B=[x_B]P$ відкритий ключ Боба
  \item $Encap(Q_B) \rightarrow (k, C)$, де $k$ - симетричний ключ для шифрування повідомлення, $C$ - інкапсульований (зашифрований) ключ, Аліса:
  \begin{enumerate}
      \item вибирає симетричний ключ $k\leftarrow_\$$ яким потім шифрує повідомлення (наприклад AES)
      \item генерує деякий ефемерний секрет $x_A$, відкритий ключ $Q_A=[x_A]P$
      \item обчислює спільний секрет з Бобом: $S=KDF([x_A]Q_B)$, де $KDF$ - функція виводу ключа, може бути взята як деяка криптографічна геш-функція
      \item інкапсулює(шифрує) симетричним алгоритмом у відповідному режимі значення $k$: $C_k = Wrap(S, k)$
      \item повертає $C=(k, (C_k, Q_A))$
  \end{enumerate}
  \item $Decap(x_B, (C_k, Q_A)) \rightarrow k$
  \begin{enumerate}
      \item знаходимо ефемерний спільний секрет з Алісою: $S=KDF([x_B]Q_A)$
      \item розшифровуємо (декапсульовуємо) та повертаємо ключ $k=Unwrap(S,C_k)$, яким розшировуємо потім повідомлення
  \end{enumerate}
  \end{itemize}
\end{frame}
\begin{frame}{Схема шифрування Ель-Гамаля}

\begin{itemize}
    \item $KeyGen(1^\lambda) \rightarrow k, Q,\ Q=[k]P$.
    \item Шифрування повідомлення \(M\):
      \[
      Enc(M, Q) = C = \bigl([r]P,\; M + [r]Q\bigr),
      \]
      де \(r \leftarrow_\$\) --- випадкове число.
    \item Розшифрування: \( M =Dec(C, k) = (M + [r]Q) - [k]([r]P)\).
    \item \textbf{Модель стійкості:} Схема є IND-CPA-безпечною, але не є IND-CCA2 стійкою, тому використовуйте з обережністю
    \item Схема є гомоморфмною: $Enc(M_1+M_2,Q)=Enc(M_1,Q)+Enc(M_2,Q)$ що дає переваги при побудові деяких криптосистем
  \end{itemize}
\end{frame}
%%%%%%%%%%%%%%%%%%%%%%%%%%%%%%%%%%%%%%%%%%%%%%%%%%%%%%%%%%%%%%%%%%%%%%
% Слайд 3: ECDSA: підпис, верифікація, моделі стійкості та недоліки
%%%%%%%%%%%%%%%%%%%%%%%%%%%%%%%%%%%%%%%%%%%%%%%%%%%%%%%%%%%%%%%%%%%%%%
\begin{frame}{Підпис за ECDSA}
  $KeyGen(1^\lambda) \rightarrow d, Q,\ Q=[d]P$

  Підпис $Sign(m, d) \rightarrow \sigma $
  \begin{enumerate}
    \item Обчислити хеш повідомлення: \(e = H(m)\), де $H$ - криптографічна геш-функція
    \item Обрати випадкове \(k\) і обчислити \(R = [k]P = (x_R, y_R) \); визначити \(r = x_R \mod n\).
    \item Обчислити \(s = k^{-1}(e + d\,r) \mod n\) (де \(d\) --- приватний ключ).
    \item Повернути підпис: \(\sigma = (r,s)\).
  \end{enumerate}

  \textbf{Верифікація $Verify(\sigma, m) \rightarrow bool$:}
  \begin{enumerate}
  \item Обчислити хеш повідомлення: \(e = H(m)\)
    \item Обчислити \(w = s^{-1} \mod n\), \(u_1 = e\,w \mod n\), \(u_2 = r\,w \mod n\).
    \item Обчислити \(R' = [u_1]P + [u_2]Q\) та перевірити \(r \equiv x_{R'} \mod n\).
  \end{enumerate}

  \textbf{Моделі стійкості:}
  \begin{itemize}
    \item ECDSA забезпечує стійкість до екзистинційних підробок підпису (EUF-CMA) в моделі з випадковим оракулом
  \end{itemize}
\end{frame}

%%%%%%%%%%%%%%%%%%%%%%%%%%%%%%%%%%%%%%%%%%%%%%%%%%%%%%%%%%%%%%%%%%%%%%
% Слайд 4: Недоліки ECDSA: малітабельність та атака на повторне використання r
%%%%%%%%%%%%%%%%%%%%%%%%%%%%%%%%%%%%%%%%%%%%%%%%%%%%%%%%%%%%%%%%%%%%%%
\begin{frame}{Недоліки ECDSA та атаки}
  \textbf{Малітабельність підпису:}
  \begin{itemize}
    \item Якщо \((r,s)\) є валідним підписом, то \((r, -s \mod n)\) також є валідним підписом для того ж повідомлення.
    \item Це створює проблему малітабельності, оскільки підпис можна «перевернути», не змінюючи повідомлення.
  \end{itemize}
  \textbf{Атака на повторне використання \(r\):}
  Нехай два підписи \((r,s_1)\) і \((r,s_2)\) на різні повідомлення $m_1, m_2$ з однаковим значенням $k$:
  \begin{align*}
  & s_1 \equiv k^{-1}(e_1 + d\,r) \pmod{n},\ s_2 \equiv k^{-1}(e_2 + d\,r) \pmod{n}.\\
  & s_1 - s_2 \equiv k^{-1}(e_1 - e_2) \pmod{n} \Rightarrow k = (e_1-e_2)/(s_1-s_2) \mod{n}
  \end{align*}
Далі зловмисник обчислює приватний ключ $d = r^{-1}(ks_1 -e_1)$

Висновок: криптосистема ECDSA дуже чутлива до якості випадковості
\end{frame}

%%%%%%%%%%%%%%%%%%%%%%%%%%%%%%%%%%%%%%%%%%%%%%%%%%%%%%%%%%%%%%%%%%%%%%
% Слайд 5: Схема ідентифікації та підпису Шнорра: основна ідея та моделі стійкості
%%%%%%%%%%%%%%%%%%%%%%%%%%%%%%%%%%%%%%%%%%%%%%%%%%%%%%%%%%%%%%%%%%%%%%
\begin{frame}{Схема ідентифікації Шнорра та підпису Шнорра}
\begin{block}{Протокол ідентифікації Шнорра}
     \textbf{} $Prover$ переконує $Verifier$ що він знає деяке $x$ не розкриваючи його
  \begin{align*}
      Prover: x, Q=[x]P & \xrightarrow{Q} Verifier:Q \\
      r \in_\$ \mathbb{Z}_n, R=[r]P & \xrightarrow{R} \\
      & \xleftarrow{c} c\in_\$ \mathbb{Z}_n \\
      s=xc+r \mod n& \xrightarrow{s} [s]P =^? R + [c]Q 
  \end{align*}
\end{block}
 

  Схема ідентифікації Шнорра є доказом знання (PoK) дискретного логарифму з нульовим розголошенням в моделі з випадковим оракулом та є найпростішим $\Sigma$-протоколом.

  Застосовуючи перетворення Фіат-Шаміра (перехід до неінтерактивного протоколу + додаємо в транскрипт повідомлення $m$, моделюємо випадковий оракул критографічною геш-функцією $H$): \(c = H(R \, \|\, m)\) отримуємо схему підпису
  
\end{frame}

\begin{frame}{Підпис Шнорра}
  \begin{itemize}
  \item $KeyGen(1^\lambda) \rightarrow (x,Q),\ Q=[x]P$
  \item $Sign(x, m) \rightarrow (R, s)$
  \begin{enumerate}
      \item Підписувач обирає випадкове $r\in_\$ \mathbb{Z}_n$ та обчислює $R=[r]P,\ e=H(R \, \|\, m)$
      \item Обчислює $s=xe + r$
      \item Публікує пару $\sigma=(e,s)$ (або альтернативно пару $\sigma = (R, s)$) як підпис повідомлення m
  \end{enumerate}
  \item $Verify(\sigma, m) \rightarrow bool$: перевіряючий обчислює $R_v = [s]P - [e]Q$ та перевіряє виконання рівності $e=^?H(R_v \, \|\, m)$ або $R_v =^? R$ в альтернативному варіанті
    \item Підпис Шнорра забезпечує сильну екзистенційну стійкість до підробок sEUF-CMA в моделі з випадковим оракулом.
    \item Підпис Шнорра дозволяє вкорочувати значення $e$, таким чином формуючи коротші підписи
    \item Дозволяє агрегувати відкриті ключі (в альтернативному варіанти) та формувати коротший підпис під одним і тим же повідомленням: $Sign(x_1, m) + Sign(x_2,m) = Sign(x_1+x_2,m)$
  \end{itemize}
\end{frame}
%%%%%%%%%%%%%%%%%%%%%%%%%%%%%%%%%%%%%%%%%%%%%%%%%%%%%%%%%%%%%%%%%%%%%%
% Слайд 6: Криптографічні комітменти на еліптичних кривих: схема Педерсена
%%%%%%%%%%%%%%%%%%%%%%%%%%%%%%%%%%%%%%%%%%%%%%%%%%%%%%%%%%%%%%%%%%%%%%
\begin{frame}{Криптографічні комітменти (Педерсен)}
  \textbf{Комітмент Педерсена:}
  \begin{itemize}
    \item Для значення \(v\) (наприклад, суми) та випадкового \(r\) обираються генератори \(G\) та \(H\).
    \item Комітмент задається як:
      \[
      C = [v]G + [r]H.
      \]
  \end{itemize}
  \vspace{0.3cm}
  \textbf{Властивості:}
  \begin{itemize}
    \item \emph{Прихованість:} Без знання \(r\) важко відновити \(v\).
    \item \emph{Обов'язковість:} Коммітер не може підмінити іншу пару $(v,r)$ при відкритті коммітменту.
    \item \emph{Гомоморфізм:} \(C_1 + C_2 = [v_1+v_2]G + [r_1+r_2]H\).
  \end{itemize}
  
\end{frame}

%%%%%%%%%%%%%%%%%%%%%%%%%%%%%%%%%%%%%%%%%%%%%%%%%%%%%%%%%%%%%%%%%%%%%%
% Слайд 7: Вибір параметрів для криптосистем на еліптичних кривих
%%%%%%%%%%%%%%%%%%%%%%%%%%%%%%%%%%%%%%%%%%%%%%%%%%%%%%%%%%%%%%%%%%%%%%
\begin{frame}{Вибір параметрів}
  \textbf{Практичні рекомендації:}
  \begin{itemize}
    \item Для рівня стійкості $1^\lambda$ слід обрати еліптичну криву з розміром простої підгрупи $2\lambda$ біт.
    \item Якщо в протоколах підпису використовується неякісне джерело випадковості слід розглянути використання детерміністичного генератора PRNG, що ініціюється значеннями $H(m), sk$, наприклад схему, що описана в RFC7969.
    \item Слід уникати слабких кривих (в яких задача DLP вирішується ефективніше ніж класичні алгоритми), наприклад для яких $n | p^r-1$, де $l$ - невелике число (так звана MOV-атака $E(F_p) \subseteq F^*_{p^r}$), багато суперсингулярних кривих також є слабкими. 
    \item \textbf{Приклади:}
      \begin{itemize}
        \item Стандарти NIST.SP800.186, ANSI X9.62, X9.63, ДСТУ4145-2002
        \item Криві від спільноти: Curve25519, Brainpool та багато інших
        \item Ось невелика база з кривими https://neuromancer.sk/std/
      \end{itemize}
  \end{itemize}

\end{frame}

%%%%%%%%%%%%%%%%%%%%%%%%%%%%%%%%%%%%%%%%%%%%%%%%%%%%%%%%%%%%%%%%%%%%%%
% Слайд 8: Підсумки лекції
%%%%%%%%%%%%%%%%%%%%%%%%%%%%%%%%%%%%%%%%%%%%%%%%%%%%%%%%%%%%%%%%%%%%%%
\begin{frame}{Підсумки лекції}
  \begin{itemize}
    \item Еліптичні криві - важливий будівельний блок дуже багатьох криптографічних протоколів.
    \item Більшість припущень стійкості класичних криптосистем зводяться до складності задачі дискретного логарифмування (ECDLP) (що в свою чергу зводиться до задачі знаходження прихованої підгрупи HSP).
    \item Еліптичні криві дають найменші ключі серед аналогічних протоколів в інших групах, достатньо працювати в 256-бітному порядку кривої для стійкості на рівні 128 біт.
    \item Окрім класичних існує ряд інших криптосистем на основі еліптичних кривих, з якими познайомимось на наступних лекціях
  \end{itemize}
\end{frame}





  \end{darkframes}




\end{document}
